% https://www.overleaf.com/learn/latex/Questions/How_can_I_get_my_table_or_figure_to_stay_where_they_are%2C_instead_of_going_to_the_next_page%3F
% \usepackage[section]{placeins}

\pgfplotsset{compat=1.15}

% Subfigures
\makeatletter
\let\MYcaption\@makecaption
\makeatother
% \usepackage[font=scriptsize]{caption}
\usepackage[font+=smaller,skip=1pt]{subcaption}
\makeatletter
\let\@makecaption\MYcaption
\makeatother

\graphicspath{{figure/}}

\tikzstyle{every picture}+=[font=\sffamily]

% https://tex.stackexchange.com/a/169711
\renewcommand{\floatpagefraction}{.9}

\newcommand{\nafig}[5]{
    % #1 postion
    % #2 subfigure width, e.g., .48\linewidth
    % #3 figure path
    % #4 label
    % #5 caption
    \begin{subfigure}[#1]{#2}
        \centering
        \includegraphics[width=\linewidth]{#3}
        \caption{\label{#4}#5}
    \end{subfigure}
}

% Top align figure and bottom align caption
% https://tex.stackexchange.com/questions/152818/top-aligned-subfigure-with-bottom-aligned-caption
\ExplSyntaxOn
\NewDocumentCommand{\xsubfigure}{ m m }
 {% #1 is a symbolic key, #2 is a list of key-value pairs
  \roly_xsubfigure:nn { #1 } { #2 }
 }
\NewDocumentCommand{\makerow}{ m }
 {% #1 is a list of symbolic keys
  \roly_makerow:n { #1 }
 }

% define the keys
\keys_define:nn { roly/subfigures }
 {
  width .tl_set:N = \l_roly_subfig_width_tl,
  body .tl_set:N = \l_roly_subfig_body_tl,
  caption .tl_set:N = \l_roly_subfig_caption_tl,
 }

% the needed variables
\dim_new:N \l_roly_row_height_dim
\box_new:N \l_roly_body_box

% this is the inner command that stores the properties
\cs_new_protected:Npn \roly_xsubfigure:nn #1 #2
 {
  \prop_if_exist:cTF { l_roly_subfig_#1_prop }
   {
    \prop_clear:c { l_roly_subfig_#1_prop }
   }
   {
    \prop_new:c { l_roly_subfig_#1_prop }
   }
  \keys_set:nn { roly/subfigures } { #2 }
  \prop_put:cnV { l_roly_subfig_#1_prop } { width } \l_roly_subfig_width_tl
  \prop_put:cnV { l_roly_subfig_#1_prop } { body } \l_roly_subfig_body_tl
  \prop_put:cnV { l_roly_subfig_#1_prop } { caption } \l_roly_subfig_caption_tl
 }

% this is the inner command for producing a row
\cs_new_protected:Npn \roly_makerow:n #1
 {
  % get the heights of the objects on a row
  \dim_zero:N \l_roly_row_height_dim
  \clist_map_inline:nn { #1 }
   {
    \hbox_set:Nn \l_roly_body_box
     {
      \prop_item:cn { l_roly_subfig_##1_prop } { body }
     }
    \dim_compare:nT { \box_ht:N \l_roly_body_box > \l_roly_row_height_dim }
     {
      \dim_set:Nn \l_roly_row_height_dim { \box_ht:N \l_roly_body_box }
     }
   }
  % produce a line
  \clist_map_inline:nn { #1 }
   {
    % a subfigure is set here
    \begin{subfigure}[t]{ \prop_item:cn { l_roly_subfig_##1_prop } { width } }
    \raggedright
    \vspace{0pt} % for top alignment
    % the body is set in a suitably dimensioned parbox
    \parbox[t][\l_roly_row_height_dim]{\textwidth}{
      \prop_item:cn { l_roly_subfig_##1_prop } { body }
    }
    % add the caption
    \caption{ \prop_get:cn { l_roly_subfig_##1_prop } { caption } }
    \end{subfigure}
    \hspace{2em} % some space between the objects in a row
   }
   \unskip\\ % end up the row
 }
\ExplSyntaxOff
